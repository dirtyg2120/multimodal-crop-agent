\chapter{Evaluation Plan and Expected Contributions}
\section{Proposed Evaluation Metrics}
The system will be evaluated using three primary metrics to ensure each part of the loop is performing correctly:
\begin{itemize}
	\item \textbf{Mean Average Precision:} Used to assess the localization accuracy of Grounding DINO in identifying leaves, bugs, and worms.
	\item \textbf{Accuracy:} Used to measure the performance of the dual-CLIP verification layer in identifying specific pathologies.
	\item \textbf{Retrieval Relevance:} Evaluates the RAG system's ability to provide the correct expert treatment advice from ChromaDB based on the identified condition.
	\item \textbf{Visual Case Analysis:} This analysis focuses on the agent's robustness in many other images from the Internet with complex context, ensuring the reasoning and recommendation still perform well.
\end{itemize}

%\chapter{Expected Contributions \& Conclusion}
\section{Expected Contributions}
This thesis aims to contribute to the field of Agentic AI in agriculture by:
\begin{itemize}
	\item Providing a solid ``Observe-Verify-Reason-Act'' framework that bridges the gap between raw vision and expert reasoning.
	\item Demonstrating the use of Open-Set Detection to handle the diverse and changing nature of farm environments without constant retraining.
	\item Developing a method for knowledge grounding that ensures AI recommendations are based on verified agricultural facts.
\end{itemize}

The primary contribution of this research is the development of a stateful agentic loop that integrates open-set detection with expert knowledge. By moving away from static pipelines, this work demonstrates a more flexible and reliable approach to precision agriculture.

\section{Future Work}
Future developments for this project include:
\begin{itemize}
	\item Image Quality Check: Adding a layer to detect motion blur, poor lighting, or extreme occlusion. If the quality is too low, the agent will request a new image rather than risk a wrong diagnosis.
	\item SAHI Integration: Combining Slicing Aided Hyper Inference (SAHI) with these quality checks to help Grounding DINO handle high-resolution images, ensuring small pests are not missed due to image downscaling.
	\item Stateful Memory: The system will save every detection result (with timestamps) into a long-term database. The agent will be able to query this history to identify patterns.
	\item Specialist Model Fallback: Implementing a multi-tier verification system for high-uncertainty cases.
\end{itemize}