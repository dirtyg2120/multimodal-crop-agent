\documentclass[twoside,a4paper,12pt]{book}
\usepackage{pdfpages}
\usepackage{tikz}
\usetikzlibrary{calc}
\usepackage{pgf}
\usepackage{etoolbox}
%\usepackage{mathptmx}
%\usepackage{fontspec}
%\setmainfont{Times New Roman}

% Layout
\usepackage{ragged2e}
\usepackage{subcaption}
\usepackage{multicol}
\usepackage{nopageno}

%%%%%%%%%%%%%%%%%%%%%%%%%%%%%%%%%%%%%%%%%%%%%%%
%\usepackage[margin=1in]{geometry}
\usepackage[left=3cm,right=2cm,top=2cm,bottom=2cm]{geometry}
\usepackage{pdflscape}
%\setlength{\parindent}{0pt}
\setlength{\parskip}{5pt}
\usepackage{indentfirst}

\makeatletter
\newcommand\mynobreakpar{\par\nobreak\@afterheading} 
\makeatother

% Symbols
\usepackage{amssymb}
\usepackage{pifont}
\newcommand{\cmark}{\ding{51}}%
\newcommand{\xmark}{\ding{55}}%
\usepackage{newtxmath}

% Heading
%%%%%%%%%%%%%%%%%%%%%%%%%%%%%%%%%%%%%%%%%%%%%%%%%%%%%%%%%%%
\usepackage[raggedright]{titlesec}
\titleformat{\paragraph}[hang]{\normalfont\normalsize\bfseries}{\theparagraph}{1em}{}
\titlespacing*{\paragraph}{0pt}{3.25ex plus 1ex minus .2ex}{0.5em}

% Float environment
%%%%%%%%%%%%%%%%%%%%%%%%%%%%%%%%%%%%%%%%%
\makeatletter
\renewcommand*{\fps@figure}{htbp}
\renewcommand*{\fps@table}{htbp}
\makeatother
\usepackage{float}
\usepackage{graphicx}
\usepackage{subcaption}
\usepackage{caption}
\captionsetup[figure]{font=small}
\usepackage[export]{adjustbox}

% Table
%%%%%%%%%%%%%%%%%%%%%%%%%%%%%%%%%%%%%
\usepackage{booktabs}
\usepackage{array}
\usepackage{tabularx}
\usepackage{makecell}
\usepackage{multirow}
\usepackage{longtable}
%\renewcommand\tabularxcolumn[1]{m{#1}}% for vertical centering text in X column
\renewcommand\theadfont{\bfseries}
\renewcommand\cellalign{tl}
\renewcommand\cellgape{\gape}
%\renewcommand{\arraystretch}{2}
\usepackage{rotating}
\usepackage{siunitx}
%\sisetup{parse-numbers = false,input-decimal-markers={-},output-decimal-marker = \text{--}}

%\usepackage{xcolor}
\usepackage{enumitem}
\setlist{nosep,topsep=0pt}

% Math
%%%%%%%%%%%%%%%%%%%%%%%%%%%%%%%%%%%%%%%%%%%%%%%
%\usepackage{commath}
\usepackage{empheq}
\usepackage{amsmath,amssymb}
\renewcommand*{\vec}[1]{#1}
\newcommand{\mat}[1]{#1}
\newcommand{\dpd}[2]{\frac{\partial{#1}}{\partial{#2}}}
\DeclareMathOperator{\MSG}{MSG}
\DeclareMathOperator{\AGG}{AGG}
\DeclareMathOperator{\SUM}{SUM}
\DeclareMathOperator{\MEAN}{MEAN}
\DeclareMathOperator{\MAX}{MAX}

% Code
%%%%%%%%%%%%%%%%%%%%%%%%%%%%%%%%%%%%%%%%%%%%%%%%%%%%%%%%
%\usepackage{algorithm}
%\usepackage{algorithmic}

\usepackage{minted}
\usemintedstyle{solarized-light}
\setminted[]{frame=single, tabsize=4, samepage}
%\setminted[rout]{
	%    bgcolor = Routbackground,
	%    framerule=0.2pt,framesep=1pt
	%}

%\usepackage{listings}
%\usepackage{xcolor}
%\definecolor{codegreen}{rgb}{0,0.6,0}
%\definecolor{codegray}{rgb}{0.5,0.5,0.5}
%\definecolor{codepurple}{rgb}{0.58,0,0.82}
%\definecolor{backcolour}{rgb}{0.95,0.95,0.92}
%\lstdefinestyle{mystyle}{
	%    backgroundcolor=\color{backcolour},   
	%    commentstyle=\color{codegreen},
	%    keywordstyle=\color{magenta},
	%    numberstyle=\tiny\color{codegray},
	%    stringstyle=\color{codepurple},
	%    basicstyle=\ttfamily\footnotesize,
	%    breakatwhitespace=false,         
	%    breaklines=true,                 
	%    captionpos=b,                    
	%    keepspaces=true,                 
	%    numbers=left,                    
	%    numbersep=5pt,                  
	%    showspaces=false,                
	%    showstringspaces=false,
	%    showtabs=false,                  
	%    tabsize=2
	%}
%\lstset{style=mystyle}

% Header and footer
%%%%%%%%%%%%%%%%%%%%%%%%%%%%%%%%%%%%%%%%%%%%%%%%%%%%%%%%%
\usepackage{lastpage}
\usepackage{fancyhdr}
\usepackage{fancybox}
\fancypagestyle{empty}{%
	\fancyhf{}%
	\renewcommand{\headrulewidth}{0pt}%
	\renewcommand{\footrulewidth}{0pt}%
	%\setlength{\headheight}{0pt}%
}
\fancypagestyle{open}{%
	\fancyhf{}%
	%\fancyfoot{} % clear all footer fields
	\fancyfoot[L]{\scriptsize \ttfamily Graduation Thesis, Semester 212, Academic year 2021 - 2022}
	\fancyfoot[R]{\scriptsize \ttfamily Page {\thepage}/\pageref{LastPage}}
	\renewcommand{\headrulewidth}{0pt}
	\renewcommand{\footrulewidth}{0.3pt}
	
}
\fancypagestyle{myfancy}{%
	\fancyhf{}
	\setlength{\headheight}{24.5pt}
	\fancyhead[L]{
		\begin{tabular}{rl}
			\begin{picture}(25,15)(0,0)
				\put(0,-8){\includegraphics[width=8mm, height=8mm]{image/HCMUT_official_logo.png}}
			\end{picture} &
			\begin{tabular}{l}
				\textbf{\bf \ttfamily Ho Chi Minh City University of Technology}\\
				\textbf{\bf \ttfamily Faculty of Computer Science and Engineering}
			\end{tabular}
		\end{tabular}
	}
	\fancyhead[R]{
		\begin{tabular}{l}
			\tiny \bf \\
			\tiny \bf
		\end{tabular}
	}
	\fancyfoot[L]{\scriptsize \ttfamily Graduation Thesis, Semester 2, Academic year 2021 - 2022}
	\fancyfoot[R]{\scriptsize \ttfamily Page {\thepage}/\pageref{LastPage}}
	\renewcommand{\headrulewidth}{0.3pt}
	\renewcommand{\footrulewidth}{0.3pt}
}

\usepackage{xurl}
% Hyperref must be loaded last!
%\PassOptionsToPackage{hyphens}{url}
\usepackage[
colorlinks=true,
allcolors=blue,
linkcolor=black,
breaklinks=true,
]{hyperref}

\begin{document}
	\frontmatter
	\begin{titlepage}
	\begin{center}
		VIETNAM NATIONAL UNIVERSITY, HO CHI MINH CITY\\
		HO CHI MINH CITY UNIVERSITY OF TECHNOLOGY\\
		DEPARTMENT OF COMPUTER SCIENCE AND ENGINEERING 
	\end{center}
	
	\vspace{1cm}
	
	\begin{figure}[h!]
		\begin{center}
			\includegraphics[width=3cm]{images/hcmut.png}
		\end{center}
	\end{figure}
	
	\vspace{1cm}
	
	\begin{center}
		\begin{tabular}{c}
			\multicolumn{1}{l}{\textbf{{\Large Graduation Thesis Proposal - Semester 211}}}\\
			~~\\
			\hline
			\\
			\textbf{\Huge Development of a Feature-Based}\\[5pt]
			\textbf{\Huge Social Bot Detection Tool }\\
			% \textbf{\Large its application on web browser extension}
			% \textbf{\Huge }\\
			\\
			\hline
		\end{tabular}
	\end{center}
	
	\vspace{3cm}
	
	\begin{table}[h]
		\begin{tabular}{rrll}
			\hspace{5 cm} 
			%& Council: & Software Engineering&\\
			& Instructor: & Assoc. Prof. Nguyen Manh Hung & \\
			& Students: & Tran Quoc Anh &--  2370386 \\
		\end{tabular}
	\end{table}
	
	\begin{center}
		{\footnotesize Ho Chi Minh City, December 2025}
	\end{center}
\end{titlepage}
	%%%%%%%%%%%%%%%%%%%%%%%%%%%%%%%%%%%%%%%%%
	\chapter*{\centering{Declaration of Authenticity}}
	We declare that this research is our own work, conducted under the supervision and guidance of Assoc. Prof. Nguyen Manh Hung. The result of our research is legitimate and has not been published in any forms prior to this. All materials used within this researched are collected ourself by various sources and are appropriately listed in the references section. \\
	
	In addition, within this research, we also used the results of several other authors and organizations. They have all been aptly referenced. \\
	
	In any case of plagiarism, we stand by our actions and will be responsible for it. Ho Chi Minh City University of Technology therefore are not responsible for any copyright infringements conducted within our research. \\[20pt]
	\begin{flushright}
		Ho Chi Minh City, December 2025 \\
		Authors \\
		Tran Quoc Anh \\
	\end{flushright}
	%%%%%%%%%%%%%%%%%%%%%%%%%%%%%%%%%%%%%%%%%
	\chapter*{\centering{Acknowledgment}}
	We would like to acknowledge and express our gratitude to everyone who have supported us during our university life. Not only the professors in Ho Chi Minh City University of Technology who have taught us immense knowledge and given us their aspiring guidance, invaluable constructive criticism and friendly advice, but also our friends and our family members of each member in our team who let us have a good physical and mental health during our study.
	
	Foremost, we would like to give our sincere gratitude to our supervisor, Associate Professor Ph.D Nguyen Manh Hung as well as our mentors for their continuous support. Their guidance has brought up our motivation and enthusiasm in research for this thesis. We could not have imagined having a better support for our study.
	
	Last but not least, we would like to thank Ho Chi Minh City University of Technology for giving us the opportunity to learn great lessons of theory and practical experience in the field of Computer Science.
	%%%%%%%%%%%%%%%%%%%%%%%%%%%%%%%%%%%%%%%%%
	\chapter*{\centering{Abstract}}
	Online social networks, such as Twitter and Facebook, have become an important part of people's daily lives in recent years since they allow them to connect with others, communicate with friends, share personal content with others, and obtain information, especially during the ongoing worldwide disaster---the COVID-19 pandemic. However, it also creates opportunities for social bots that are designed to replicate the behaviors of normal human accounts. Most of these bots are used for nefarious purposes such as disseminating false information, artificially amplifying the popularity of a person or movement or spreading spam. This proposal presents a review of some techniques that have emerged and designed to differentiate between social bot accounts and human accounts. We restrict the analysis to the detection of social bots on the Twitter social media platform for the time being, with plans to expand to Facebook later. We also compare the experiments of each technique on two different data sets to point out the most accurate one. Besides experimenting, our final destination is to develop a social bot detection tool that can be deployed as a web application. Finally, we highlight the challenges that remain in the domain of social bots detection and consider future directions for research efforts that are designed to address this problem.
	
	%%%%%%%%%%%%%%%%%%%%%%%%%%%%%%%%%%%%%%%%%
	\tableofcontents
	\thispagestyle{empty}
	%%%%%%%%%%%%%%%%%%%%%%%%%%%%%%%%%%%%%%%%%
	\listoffigures
	\thispagestyle{empty}
	%%%%%%%%%%%%%%%%%%%%%%%%%%%%%%%%%%%%%%%%%
	\listoftables
	\thispagestyle{empty}
	%%%%%%%%%%%%%%%%%%%%%%%%%%%%%%%%%%%%%%%%%
	\mainmatter
	\pagestyle{myfancy}
	\begingroup
	\chapter{Introduction}
\label{chap:intro}
\thispagestyle{open}

	\chapter{Theoretical Background}
\label{chap:theory}
\thispagestyle{open}

\section{Deep Learning in Computer Vision}
\subsection{CNNs to Transformers}
% Since the success of AlexNet in 2012\cite{russakovsky2015ilsvrc}
Since 2012, Convolutional Neural Networks (CNNs) have formed the backbone of most computer vision systems. Successive architectures, including VGG\cite{simonyan2014vgg}, ResNet\cite{he2016resnet}, and EfficientNet\cite{tan2019efficientnet}, consistently achieved strong performance on large-scale benchmarks such as ImageNet\cite{russakovsky2015ilsvrc} and COCO\cite{lin2014coco}. Besides that, CNN-based detectors like YOLO\cite{redmon2016yolo} became widely adopted in real-time applications due to their computational efficiency. CNNs are built to focus on local patterns and recognize them anywhere, which makes them efficient, lightweight, and ideal for running on edge devices.

For tasks requiring global context, this same locality bias is a bottleneck. Capturing long-range dependencies---such as spatial relationships between distant plant organs or complex scene configurations---often requires very deep CNN architectures or additional fine-tuned modules. These designs increase model complexity and may still be insufficient for fine-grained recognition tasks where entire scene understanding is essential.

The introduction of Vision Transformers (ViTs)\cite{dosovitskiy2021vit} marked a fundamental change in computer vision . ViTs model global interactions from the first layer by using self-attention mechanisms and treating images as sequences of patches. Unlike the hierarchical aggregation used in CNNs, the self-attention mechanism allows every patch to attend to every other patch, which is particularly advantageous for capturing long-range dependencies in complex scenes.

There are two primary implications for modern vision systems: \begin{itemize} 
	\item \textbf{Global Reasoning:} ViTs are better able to handle occlusion and complex spatial distributions due to their natural integration of global structure.
	\item \textbf{Multimodal Alignment:} Since Transformers process image patches and text tokens via the same attention mechanism, they make it easier to create unified embedding spaces . This is the foundation of models like CLIP, enabling the zero-shot and open-vocabulary capabilities required for adaptable crop health management in open-set agricultural environments.
\end{itemize}

%%%%%%%%%%%%%%%%%%%%%%%%%

\subsection{Closed-Set vs. Open-Set Detection}

Conventional object detectors, from two-stage models like Faster R-CNN\cite{fasterrcnn_ren2015} to one-stage architectures like YOLO, typically operate in a closed-set setting. These models learn a mapping from an image to a fixed, finite set of categories $\mathcal{Y}_{\text{known}}$ (e.g., the 80 classes in MS COCO)\cite{lin2014coco}. At test time, the model implicitly assumes that all visible objects belong to this known set. Any object outside this taxonomy - such as a rare pest or a novel disease symptom - is either forced into a known class (misclassification) or treated as background (suppression).

\emph{Open-set object detection} (OSOD) generalizes this formulation. It requires a model to (i) accurately classify instances from $\mathcal{Y}_{\text{known}}$, while (ii) acknowledging inputs from $\mathcal{Y}_{\text{unknown}}$ without overconfident misclassification.\cite{dhamija2020openset} While early OSOD work focused on rejecting unknowns, recent advances in \emph{open-vocabulary} models provide a more flexible solution. By leveraging large-scale vision--language pre-training (e.g., CLIP), these models replace fixed classification layers with text-embedding similarity scores.\cite{radford2021clip,zhu2024surveyovd} This allows a detector not just to reject an unknown object, but to localize it based on descriptive text prompts (e.g.,``necrotic lesion with yellow halo'') that were never seen during training.

This distinction is critical because agriculture is inherently an \emph{open-set} environment. The biological landscape is dynamic: new pathogens emerge, invasive species migrate, and symptom appearances shift due to environmental stress.\cite{fuentes2021opensettomato} A closed-set detector is brittle in this context; it cannot recognize a threat it has not been explicitly trained to see. Recent work in plant pathology underlines the necessity of models that can handle these ``unknown'' classes to prevent silent failures in the field.\cite{shen2024opensetpest}

In this thesis, we address this limitation by operationalizing crop health monitoring as a \emph{text-conditioned, open-set} problem. Rather than relying on a static list of pest categories, we propose a pipeline that integrates open-vocabulary detection with agronomic knowledge. The system is designed to localize both known and novel visual patterns by conditioning the detector on textual descriptions, thereby bridging the gap between rigid closed-set training and the dynamic reality of agricultural management.

%%%%%%%%%%%%%%%%%%%%%%%%%

\section{Grounding DINO}
Grounding-DINO\cite{liu2023grounding} is a popular large vision-language model equipped with open-set object detection capabilities. It is designed to overcome the constraints of fixed-class detectors by recognizing novel objects defined solely by natural language input. The model achieves this by combining the high-performance transformer-based detection architecture of DINO with the grounded language-image pre-training methodology of GLIP. This integration allows Grounding DINO to perform zero-shot detection by aligning visual features with textual descriptions at a foundational level.
\subsection{DETR}
DETR (Detection Transformer) \cite{DBLP:journals/corr/abs-2005-12872} is a deep learning model that fundamentally shifts object detection from a regression and classification task to a direct set prediction problem. By combining the Transformer architecture with self-attention mechanisms, DETR captures global context and long-range dependencies across the image, addressing limitations of CNN-based methods that primarily focus on local features. Furthermore, its set-based prediction mechanism utilizing bipartite matching eliminates the need for hand-crafted anchors and complex post-processing steps like Non-Maximum Suppression (NMS), streamlining the pipeline into an end-to-end process.

\begin{figure}[h]
	\centering
	\includegraphics[width=\textwidth]{images/detr.png}
	\caption{DETR architecture \cite{DBLP:journals/corr/abs-2005-12872}.}
	\label{fig:detr}
\end{figure}

To achieve this, the DETR architecture follows a streamlined pipeline:
\begin{itemize}
	\item \textbf{Backbone:} The input image is processed by a CNN to extract a low-resolution feature map. Positional encodings are added to these features to preserve the spatial structure that Transformers otherwise ignore.
	\item \textbf{Encoder:} The flattened features are passed through a Transformer Encoder. Here, self-attention layers aggregate information from across the entire image, enriching each pixel's representation with global context.
	\item \textbf{Decoder:} The Decoder utilizes learned object queries - vectors representing potential objects - to interact with the encoded image features. It reasons about the relations between objects and the global image context.
	\item \textbf{Prediction Heads:} Finally, the decoder output is passed to Feed-Forward Networks (FFNs) that predict class labels and bounding box coordinates for each query. If no object is detected for a specific query, it is assigned a "no object" class.
\end{itemize}

Despite its conceptual advantages, the vanilla DETR architecture faces significant limitations, specifically regarding slow convergence rates and difficulties in processing high-resolution images.
\subsection{DINO}
\begin{figure}[h]
	\centering
	\includegraphics[width=\textwidth]{images/dino.png}
	\caption{DINO Architecture (Source: Internet).}
	\label{fig:dino}
\end{figure}
DINO (DETR with Improved deNoising anchOr boxes) builds upon the standard DETR framework by integrating specific architectural enhancements designed to optimize convergence speed and detection performance on high-resolution inputs. These improvements focus on two primary areas: attention efficiency and query stability.

\textbf{Multi-Scale Deformable Attention} \\
To process high-resolution feature maps efficiently, DINO adopts the Deformable Attention mechanism from Deformable DETR. Unlike global self-attention, which computes dependencies across all spatial locations resulting in quadratic computational complexity - deformable attention restricts the attention field to a small set of learned sampling points around a reference. This sparse attention mechanism allows the model to capture fine-grained details for small object detection without incurring prohibitive computational costs.

\textbf{Query Optimization and Denoising} \\
DINO mitigates the convergence latency of the original DETR through two complementary strategies:
\begin{itemize}
	\item \textbf{Dynamic Anchor Boxes (DAB-DETR):} Static object queries are replaced with dynamic anchor boxes. These anchors explicitly formulate the query position as dynamic spatial priors (coordinates and scale), clarifying the spatial relationship between the query and the target features.
	\item \textbf{Contrastive Denoising Training (DN-DETR):} To stabilize the bipartite matching process, DINO introduces a denoising training task. The model is fed "noisy" ground-truth bounding boxes and trained to reconstruct the original coordinates. This auxiliary task reduces the ambiguity of bipartite matching during early training phases, significantly accelerating convergence.
\end{itemize}
\subsection{GLIP and Grounded Pre-Training}
Grounding DINO extends the capabilities of standard detection by incorporating language supervision, enabling ``zero-shot'' detection where the model can identify objects based on arbitrary text descriptions without re-training. This relies on the methodology introduced by GLIP, which reformulates object detection as a \textbf{phrase grounding} task, replacing fixed class IDs with alignable word embeddings.

The model's generalization capability stems from a diverse combination of datasets during pre-training:
\begin{itemize}
	\item \textbf{Detection Data (e.g., Object365, LVIS):} Provides standard bounding boxes and class labels to maintain localization precision.
	\item \textbf{Caption Data (e.g., Cap24M):} Offers detailed textual descriptions of images to expand the semantic vocabulary.
	\item \textbf{Grounded Data (e.g., RefCOCO):} Crucially, this data links specific phrases in a sentence to specific image regions. This bridges the gap between visual features and linguistic semantics, allowing the model to learn the association between textual concepts and visual patterns directly.
\end{itemize}
\subsection{Grounding DINO Architecture}
The Grounding DINO architecture is designed to process image-text pairs simultaneously to achieve open-set detection. As illustrated in Figure \ref{fig:grounding_dino}, the model comprises several key components: a dual-backbone feature extractor, a feature enhancer for multi-modal fusion, a language-guided query selection module, and a cross-modality decoder.

\begin{figure}[h]
	\centering
	\includegraphics[width=\textwidth]{images/grounding_dino_arch.png}
	\caption{The framework of Grounding DINO \cite{liu2023grounding}.}
	\label{fig:grounding_dino_arch}
\end{figure}

\textbf{Feature Extraction and Encoding} \\
The model processes the input image and text prompt through separate backbones:
\begin{itemize}
	\item \textbf{Image Backbone:} A Swin Transformer \cite{liu2021swin} is employed to extract hierarchical visual features from different layers, capturing multi-scale visual information.
	\item \textbf{Text Backbone:} The text prompt is tokenized using Byte-Pair Encoding (BPE) and encoded by a BERT model \cite{devlin2018bert}. This produces a sequence of text embeddings with dimension $N \times 768$, where $N$ is the number of tokens.
\end{itemize}

\textbf{Feature Enhancer Module} \\
To bridge the gap between modalities, the raw image and text embeddings are fed into a Feature Enhancer. This module consists of multiple layers employing both self-attention and bi-directional cross-attention mechanisms. These layers facilitate deep interaction between text-to-image and image-to-text contexts, ensuring that the visual features are contextually aware of the linguistic prompts before detection begins.

\textbf{Language-Guided Query Selection} \\
Unlike standard DETR which uses learned static queries, Grounding DINO initializes object queries based on the input text. The model calculates the dot product between the enhanced text and image features to generate similarity scores. It then selects the top $N_I$ image features with the highest scores to serve as the initial object queries. This ensures the queries are relevant to the specific objects mentioned in the text prompt.

\textbf{Sub-Sentence Level Representation} \\
Grounding DINO refines how text is processed by introducing a ``sub-sentence'' level representation. Previous methods either compressed an entire sentence into a single feature (losing fine-grained detail) or allowed all words to interact indiscriminately, which created unnecessary dependencies between unrelated category names. To solve this, Grounding DINO employs attention masks within the text encoder. These masks selectively block the attention mechanism, preventing unrelated categories from influencing each other. This ensures the model retains precise, per-word features necessary for fine-grained understanding without introducing noise from irrelevant word associations.
\begin{figure}[h]
	\centering
	\includegraphics[width=0.9\textwidth]{images/text_feature.png}
	\caption{Comparisons of text representations  \cite{liu2023grounding}.}
	\label{fig:text_feature}
\end{figure}

\textbf{Cross-Modality Decoder and Training Objectives} \\
The selected queries are fed into a cross-modality decoder. Here, query features undergo self-attention followed by cross-attention with both the image and text feature maps to refine the object representations.
\begin{itemize}
	\item \textbf{Prediction:} The output queries predict bounding boxes and class probabilities based on their similarity to the text features.
	\item \textbf{Loss Functions:} Training optimizes a combination of losses using bipartite matching. Classification utilizes a \textbf{contrastive loss}, where the dot product between query and text features produces logits optimized via Focal Loss \cite{lin2017focal}. Bounding box regression is supervised using L1 loss and Generalized IoU (GIoU) loss \cite{rezatofighi2019giou}.
\end{itemize}

%Grounding-DINO is a popular large vision-language model equipped with open-set object detection capabilities. This enables the recognition of novel objects outside the initially defined categories.
%As shown in Figure 2, the Grounding-DINO model processes an image along with a text prompt corresponding to each class. The model comprises several key components:
%an image backbone that extracts image features, a text backbone that encodes textual information into feature vectors, a feature enhancer module that fuses these image and text features, a language-guided query selection module that initializes object queries based on the input text prompt, and finally, a cross-modality decoder that refines object features and bounding boxes.
%For each (image, text) pair, the model processes images using the Swin Transformer[35] and text via a BERT encoder [13]. Image features are hierarchically extracted from different layers of the Swin Transformer, capturing hierarchical visual information. Text is tokenized using the byte-pair encoding (BPE) scheme [49], which is then encoded by the BERT model to produce N ×768-dimensional text embeddings (features), where N represents the number of tokens in the text prompt. These raw image and text embeddings are fed into the feature enhancer module to enable cross-modal fusion. The module comprises multiple enhancer layers, each containing self-attention mechanisms for both image and text processing, as well as cross attention layers that facilitate interactions between text-to-image and image-to-text contexts.
%Grounding-DINO method uses the enhanced text features to select object queries by calculating the dot product between text and image features. It then selects NI object queries by choosing image features with the maximum scores. These language-guided queries are subse- quently fed into a cross-modality decoder. In this decoder, query features undergo a series of self-attention operations, followed by cross-attentions with both image and text features. Finally, the output queries from the decoder’s last layer are utilized to predict object boxes and corresponding class probabilities based on similarity with the text features. For training, the model uses a contrastive loss between predicted objects and language token features. To compute this loss, each query calculates a dot product with text features to produce logits for every text token. Then the focal loss [31] is applied to these logits to obtain the classification loss. Additionally, L1 and GIoU[45] losses are employed for the bounding box regression. Similar to DETR-like models [7], bipartite matching is used to find the matching between predicted and ground-truth objects. The total loss is then computed using the combination of classification and bounding-box losses based on this mapping.

%%%%%%%%%%%%%%%%%%%%%%%%%

\section{CLIP}
CLIP is a neural network adept at grasping visual concepts through natural language supervision. It operates by concurrently training a text encoder and an image encoder, focusing on a pre-training task that involves matching captions with corresponding images. This architecture allows CLIP to adapt to a variety of visual classification benchmarks seamlessly. It does so by simply receiving the names of the visual categories to be recognized.
\subsection{Learning from Unstructured Data}
Traditional computer vision models have depended on supervised learning methods that need high-quality, human-annotated datasets. In standard benchmarks, models are trained to predict a fixed set of predetermined categories, which are manually verified for accuracy. While effective for closed-set tasks, this approach has two limitations: the high cost of scaling manual annotation to new domains and the inability of the model to generalize beyond its training taxonomy.

CLIP resolves these limitations by changing from crowd-sourced labeling to natural language supervision. Instead of relying on a restricted ontology of categorical labels , CLIP learns from a massive dataset of 400 million image-text pairs collected from the public internet (WebImageText). 

This method makes use of the semantic richness of unstructured text found on the web. By treating natural language descriptions as labels, the model is not confined to a static list of IDs but instead learns to associate visual features with a vast array of linguistic concepts. This transition enables the model to generalize unseen categories without task-specific retraining.

\subsection{Contrastive Learning}
At the core of CLIP is a contrastive learning objective designed to align visual and textual representations into a shared embedding space. Unlike traditional classification tasks that map images to discrete labels, CLIP solves a proxy task of predicting which text description matches which image within a given batch.

For a batch of $N$ (image, text) pairs, the model constructs an $N \times N$ similarity matrix by computing the cosine similarity between all image and text embeddings. The training objective is to maximize the similarity scores of the $N$ correct pairings (the diagonal of the matrix) while minimizing the scores of the $N^2 - N$ incorrect pairings (the off-diagonal elements). This forces the model to learn a multi-modal embedding space where semantically similar images and texts are located close to each other.

This alignment is optimized using a symmetric cross-entropy loss over the similarity scores. Formally, for an image embedding $I_i$ and a text embedding $T_i$, the loss function for the image-to-text direction is defined as:

\begin{equation}
	\mathcal{L}_{i}^{(I \rightarrow T)} = -\log \frac{\exp(\text{sim}(I_i, T_i) / \tau)}{\sum_{j=1}^{N} \exp(\text{sim}(I_i, T_j) / \tau)}
\end{equation}

where $\tau$ is a learnable temperature parameter that scales the logits. A symmetric loss $\mathcal{L}^{(T \rightarrow I)}$ is calculated for the text-to-image direction, and the total loss is the average of these two components.
\subsection{Dual Encoders}
\begin{figure}[h]
	\centering
	\includegraphics[width=0.9\textwidth]{images/dual_encoder.png}
	\caption{Text Encoder \& Image Encoder (Source: Internet).}
	\label{fig:dual_encoder}
\end{figure}
CLIP processes two independent neural networks to process visual and textual inputs in parallel as show in Figure \ref{fig:dual_encoder}. 
\begin{itemize}
	\item \textbf{Image Encoder} (typically a Vision Transformer like ViT-B/32) maps images to feature vectors
	\item \textbf{Text Encoder} (a standard Transformer) processes natural language prompts.
\end{itemize}
Both encoders must project their outputs into a shared embedding space of identical dimension $d$ via a learned linear layer. This dimensional alignment is important, as it allows the vectors from these different modalities to be compared directly using the dot product, enabling the contrastive mechanism described previously.

\subsection{Zero-Shot Inference}

Zero-shot inference enables the model to predict classes that were not explicitly encountered during training. This is achieved by synthesizing a linear classifier directly from natural language descriptions.

At inference time, the names of the target categories (e.g., "healthy", "early blight") are inserted into a prompt template, such as "A photo of a \{label\}." These prompts are processed by the Text Encoder to generate a set of class embeddings $\{T_1, T_2, \dots, T_K\}$. Simultaneously, the input image is mapped to a feature vector $I$ by the Image Encoder.

Classification is performed by computing the cosine similarity between the image vector $I$ and the set of text vectors. The predicted class is simply the one whose text embedding maximizes the dot product with the image embedding:

\begin{equation}
	\hat{y} = \arg\max_{k} (\text{sim}(I, T_k))
\end{equation}

This mechanism effectively treats the text embeddings as the dynamic weights of a classifier, allowing the model to adapt to any open-set taxonomy solely by modifying the input text prompts.

%%%%%%%%%%%%%%%%%%%%%%%%%

\section{Large Language Models (LLMs) \& Reasoning}
\subsection{Generative Transformer Architecture}
Large Language Models (LLMs) are defined as transformer-based models scaled to billions of parameters. Unlike earlier architectures that process an entire sentence at once (like BERT), modern LLMs primarily use a decoder-only architecture. This design processes text sequentially-reading from left to right-to generate coherent responses token by token.

\textbf{Pre-training}

Next Token Prediction The fundamental "engine" of these models is built during a massive pre-training phase. The model reads vast amounts of text and learns to solve a single objective: Next Token Prediction. Here is how it works by steps:
\begin{enumerate}
	\item Given a sequence of words (e.g., "The crop leaves are turns..."), the model calculates the probability of the most likely next word (e.g., "yellow").
	\item By repeating this billions of times, the model develops a deep statistical understanding of language, grammar, and facts.
\end{enumerate}

\textbf{Instruction Tuning \& Alignment}

While pre-training teaches the model how to speak, it does not teach it to follow orders. A raw pre-trained model might just continue a sentence rather than answer a question. To fix this, modern models undergo a second training phase called Instruction Tuning (Ouyang et al., 2022). This ``fine-tuning'' step teaches the model to understand user intents and generate helpful, safe responses rather than just random text completions.

\subsection{Chain-of-Thought (CoT) Reasoning}
%ref: https://www.ibm.com/think/topics/chain-of-thoughts#f1
Chain of thought (CoT) \cite{CoT} is a prompt engineering technique that enhances the output of large language models (LLMs), particularly for complex tasks involving multistep reasoning. It facilitates problem-solving by guiding the model through a step-by-step reasoning process by using a coherent series of logical steps. Instead of giving a final answer immediately, the model generates middle reasoning steps. This process allows the model to think more, breaking down a complex problem into smaller, clearer parts before concluding.

Reasoning is often considered an ``emergent ability'', meaning it appears naturally as the size and complexity of a model increase. Larger models tend to perform better at this because they have learned more patterns from massive datasets. However, increasing model size is not the only way to improve this. Advances in "instruction tuning" now allow smaller models to perform CoT reasoning effectively by training them on examples of how to respond step-by-step.

%In this report, CoT is applied during the agent's ``Verify'' phase. Rather than allowing the model to guess a crop diagnosis instantly, the system prompts it to list its observations first. By following this logical chain-identifying symptoms first, then comparing them to data-the agent produces more accurate and reliable results.
\subsection{Retrieval-Augmented Generation (RAG)}
Retrieval-Augmented Generation \cite{RAG} is an artificial intelligence (AI) application that connects a generative AI model with an external knowledge base. The data in the knowledge base augments user queries with more context so the LLM can generate more accurate responses. RAG enables LLMs to be more accurate in domain-specific contexts without needing fine-tuning.

\textbf{Vector Embeddings} 

To retrieve this information, the system must first understand the "meaning" of the text. This is done using Vector Embeddings, which map text into a high-dimensional numerical space. For example, OpenAI’s text-embedding-3-small model converts a sentence into a vector of 1,536 numbers. In this space, concepts that are semantically similar (like ``yellow leaves'' and ``nitrogen deficiency'') are located close to each other, even if they do not share the exact same words.

\textbf{Vector Indexing}

Finding the closest matching vector among millions of documents can be slow if done one by one. To solve this, vector databases (like ChromaDB) use an algorithm called Hierarchical Navigable Small World (HNSW)\cite{hnsw}, HNSW organizes data into a multi-layered graph structure. This allows the system to quickly zoom in on the relevant neighborhood of data points rather than scanning the entire database, ensuring the agent retrieves information almost instantly.

\textbf{Semantic Search Metrics}

The relevance of a document is measured using Cosine Similarity. This metric calculates the angle between the query vector and the document vector. If the score is close to 1, it means the documents have almost the same meaning; if it is close to 0, it means they are unrelated. The agent uses this score to rank the retrieved documents and select only the most relevant evidence to verify its diagnosis.

%%%%%%%%%%%%%%%%%%%%%%%%%

\section{Agentic AI Frameworks}

\subsection{The ReAct Paradigm}
%https://www.k2view.com/blog/react-agent-llm/#What-does-ReACT-stand-for 
ReAct stands for Reasoning and Acting. It is a paradigm that overcomes the limitations of static Large Language Models (LLMs) by interleaving reasoning traces with executable actions.

In a standard LLM interaction, the model maps an input directly to an output based solely on pre-trained probability weights. In contrast, a ReAct agent operates dynamically by maintaining a continuous feedback loop:
\begin{itemize}
	\item \textbf{Thought (Reasoning):} The agent utilizes Chain-of-Thought (CoT) prompting to decompose a complex user request into logical intermediate steps. This internal monologue allows the model to plan before acting.
	\item \textbf{Action:} Instead of relying on internal parametric knowledge (which may be hallucinated), the agent invokes external tools or functions-such as querying a sensor database or calculating a specific crop metric-to retrieve ground-truth data.
	\item \textbf{Observation:} The agent perceives the output from the tool and appends this new information to its context window to formulate the next step.
\end{itemize}
%Distinction from Conversational Chatbots 
The fundamental difference between a standard chatbot and a ReAct agent lies in their operational scope. A chatbot is designed to "Talk," utilizing patterns to simulate conversation. A ReAct agent is designed to "Do," perceiving the environment through observations and affecting it through actions. 
%While a chatbot might estimate a crop's condition based on general training data, a ReAct agent executes a query to the farm’s specific sensors, observes the real-time moisture levels, and provides a recommendation grounded in that specific reality.
\subsection{Neuro-Symbolic Integration (Structured Outputs)}
While Large Language Models (LLMs) excel at probabilistic generation, they inherently lack deterministic control. In agricultural and industrial applications, this stochastic nature presents a safety risk; a system cannot ``hallucinate'' a chemical dosage or a sensor ID. To mitigate this, the system employs a Neuro-Symbolic approach.

Neuro-Symbolic AI integrates two distinct paradigms:
\begin{itemize}
	\item Neural Component: The LLM, which handles intuition, semantic understanding, and unstructured data %(e.g., "The leaf looks yellowish").
	\item Symbolic Component: A logic-based validator that enforces strict rules, types, and constraints %(e.g., Dosage must be an integer between 0 and 100).
\end{itemize}
In this thesis, this integration is implemented via Schema-Guided Generation (technically realized using Pydantic). By forcing the LLM to collapse its probability distribution into a pre-defined JSON schema, we ensure that the output is not merely text, but a valid executable object. This acts as a guardrail, ensuring that even if the agent’s reasoning is creative, its final actuation commands remain syntactically and semantically valid.
%\subsection{Control Flow: Workflows vs. Agents}
%Recent developments in agentic engineering, notably categorized by Ng (2024) and frameworks like LangChain, distinguish between two primary control architectures: Agentic Workflows and Autonomous Agents.
%
%Agentic Workflows: These rely on predefined, directed acyclic graphs (DAGs). The path of execution is hard-coded by the engineer (e.g., Step A always leads to Step B). This ensures reliability but limits flexibility; the system cannot handle edge cases outside its design.
%
%Autonomous Agents: These utilize a dynamic control loop (such as the ReAct loop). The LLM itself acts as the router, deciding which step to take next based on real-time observations. This maximizes flexibility but introduces unpredictability.
%
%Hybrid Approach This thesis proposes a hybrid architecture. The high-level orchestration follows a strict Workflow (Observe → Verify → Reason → Act) to ensure the system consistently adheres to agricultural protocols. However, within each node of this workflow, the system operates as an Autonomous Agent, capable of dynamic tool usage to satisfy the requirements of that specific step. This balances the need for industrial reliability with the adaptability required for biological environments.
	\chapter{Related Works}
\label{chap:works}
\thispagestyle{open}
\section{The Big Picture of AI in Crop Health Management}

The application of Artificial Intelligence in agriculture has evolved through three distinct paradigms, progressing from manual statistical analysis to the current use of generative reasoning. Recent surveys by Liakos et al. \cite{liakos2018machine} and Kamilaris and Prenafeta-Boldú \cite{kamilaris2018deep} classify these techniques based on their structural complexity and level of independence. Initially, techniques depended on fixed, manually created rules; however, the field has gradually moved towards data-driven Deep Learning, and most recently, toward Foundation Models that allow agents to do the reasoning rather than just classify.

\subsection{Traditional Machine Learning Approaches}
Prior to the widespread adoption of Deep Learning, crop health monitoring was dominated by feature engineering. In this paradigm, domain experts manually defined the visual characteristics of a disease-such as specific color histograms, texture patterns (e.g., GLCM), or shape descriptors. These features were then processed by classical classifiers like Support Vector Machines (SVM) or Random Forests \cite{liakos2018machine}.

While pioneers like Pydipati et al. \cite{pydipati2006identification} demonstrated success in controlled environments, these systems lacked robustness. They struggled to generalize in real-world conditions where lighting varies, or when symptoms appear subtly different across plant varieties. The rigidity of manual feature extraction meant that the system could only detect what it was explicitly programmed to look for, creating a bottleneck in scalability.

\subsection{Supervised Deep Learning Dominance}
The introduction of Convolutional Neural Networks (CNNs) marked a pivotal shift, allowing systems to automatically learn features from raw data. Early works by Mohanty et al. \cite{mohanty2016using} and Ferentinos \cite{ferentinos2018deep} utilized architectures like AlexNet and GoogLeNet on the PlantVillage dataset, achieving human-level accuracy in disease classification.

This era evolved rapidly from simple classification (labelling an entire image) to real-time object detection, often referred to as the ``YOLO Era''. Single-stage detectors such as YOLO (You Only Look Once) became the industry standard for field scouting, enabling the localization of specific pests or lesions in milliseconds \cite{li2021comprehensive}. However, these systems operate on a ``closed-set'' assumption: they can only detect the specific classes they were trained on. If a supervised model encounters a novel pest or an undocumented disease variant, it will either miss it entirely or confidently misclassify it, highlighting a critical limitation in open-field adaptability.

\subsection{The Emergence of Foundation Models and Agentic AI}
The current frontier addresses the limitations of closed-set supervision through \textbf{Foundation Models}. Unlike their predecessors, models like CLIP \cite{radford2021learning} and Grounding DINO \cite{liu2023grounding} are trained on massive, internet-scale datasets of image-text pairs. This enables ``Zero-Shot'' detection, where a system can identify a pest it has never seen before, simply by understanding its textual description.

Furthermore, the integration of Large Language Models (LLMs) has given rise to \textbf{Agentic AI}. Rather than acting as passive detectors, these systems function as autonomous agents capable of "Active Perception"-observing a symptom, verifying it, and retrieving external knowledge (RAG) to form a diagnosis. This shift moves the goalpost from simple detection accuracy to holistic, actionable decision support.


\section{Problem Definitions}

To properly contextualize the challenges in autonomous crop health management, we formally define the technical limitations inherent in traditional supervised learning and generation tasks. These definitions establish the necessity for the open-set and retrieval-augmented architecture proposed in this thesis.

\subsection{The Open-Set Recognition Problem}
Standard object detection systems operate under a ``Closed-Set'' assumption. In this paradigm, a model is trained on a dataset $D_{train} = \{(x_i, y_i)\}_{i=1}^N$, where $x_i$ is an input image and $y_i$ belongs to a fixed set of predefined classes $C_{train} = \{c_1, c_2, \dots, c_k\}$. The detection function $f$ is optimized to map an input to one of these known classes:
\[ f(x) \rightarrow \hat{y} \in C_{train} \]

The fundamental problem arises during deployment when the system encounters a novel pest or disease instance $x_{new}$ belonging to a class $c_{new}$ such that $c_{new} \notin C_{train}$. A closed-set detector is forced to map $x_{new}$ to the closest existing class in $C_{train}$, resulting in a confident error or false positive.

\textbf{Open-Set Recognition}, therefore, requires a model capable of zero-shot transfer. This is defined as learning a function $f(x, t) \rightarrow \hat{y}$, where $t$ is a natural language description, allowing the identification of classes where $C_{test} \not\subseteq C_{train}$.

\subsection{The Semantic Gap and Hallucination}
While visual detectors identify spatial features (e.g., bounding boxes), they lack the capacity to reason about causal implications. Conversely, Large Language Models (LLMs) possess general reasoning capabilities but lack visual grounding. We define the \textbf{Semantic Gap} as the disconnect between the visual feature representation $V$ (e.g., pixel patterns of a yellow leaf) and the domain-specific knowledge $K$ required to diagnose the cause (e.g., distinguishing between nitrogen deficiency and early blight).

Relying solely on generative models to bridge this gap introduces the problem of \textbf{Hallucination}. Given a user query $Q$ and a visual context $I$, a standard LLM generates a response $\hat{y}$ by maximizing the probability of the next token based on training correlations rather than factual correctness:
\[ \hat{y} = \arg\max_y P(y | I, Q) \]

Without external grounding, the model may generate plausible but factually incorrect agronomic advice. This necessitates a retrieval mechanism $R$ to condition the generation on verified documents $D$, altering the objective to maximize $P(y | I, Q, R(Q))$, ensuring the output is grounded in retrieved evidence rather than parametric memory alone.

%3.2.1 The Data Scarcity Problem: High cost of annotation for rare pests.
%
%3.2.2 The Small Object Problem: Difficulty detecting insects in wide-field images.
%
%3.2.3 The Semantic Gap: The disconnect between detecting a "spot" and knowing it is "Early Blight."
\section{Review of Single-Modality Approaches}

Most research in agricultural AI focuses on either only images or only text. While these models work well for specific tasks, they have several problems when used for automatic crop management in real fields.

\subsection{Limitations of Vision-Only Supervised Detectors}
Supervised deep learning models, particularly CNNs and the YOLO family, have been the cornerstone of plant disease detection \cite{kamilaris2018deep, ferentinos2018deep}. However, these models exhibit significant technical drawbacks when deployed outside of curated datasets:

\begin{itemize}
	\item \textbf{Data Bias and Overfitting:} Mohanty et al. \cite{mohanty2016using} demonstrated high accuracy on the PlantVillage dataset, but subsequent studies by Barbedo \cite{barbedo2018impact} found that these models rely on "shallow" features like leaf shape or background color rather than the actual pathology. When the background shifts from a laboratory setting to a complex field, performance degrades significantly.
	\item \textbf{The Feature Dependency Gap:} In supervised learning, a classifier $f(x)$ learns a mapping from high-dimensional pixels to a class index $y \in \{0, \dots, N\}$. Because the model lacks a semantic understanding of the label, it cannot distinguish between a "Bacterial Spot" and a visually similar artifact (e.g., a mud splash or shadow) unless it was explicitly trained on those "negative" examples.
	\item \textbf{High Cost of Annotation:} Annotation engineering is labor-intensive. Manually labeling thousands of bounding boxes for every new pest strain is a non-scalable process for global agronomy.
\end{itemize}

\subsection{Limitations of Text-Only Large Language Models}
While LLMs have shown sophisticated reasoning for agronomic decision support, they operate in a state of "Visual Blindness." The technical limitations include:

\begin{itemize}
	\item \textbf{Subjectivity in Symptom Description:} Without a vision-language alignment, the system relies on human-provided text descriptions. As noted in recent evaluations \cite{kasneci2023chatgpt}, this introduces a "Semantic Gap" where the user's subjective description of a leaf ("yellow spots") may not align with the scientific definition required for an accurate diagnosis.
	\item \textbf{Lack of Spatial Grounding:} Text-only models cannot localize an infection. In a field scenario, knowing \textit{what} a disease is is insufficient without knowing \textit{where} it is and its \textit{severity density}. This lack of spatial context prevents LLMs from being used for localized spray planning or robotic intervention.
	\item \textbf{Hallucination Risk:} LLMs generate responses based on token probabilities $P(y|Q)$ rather than physical evidence. In agricultural contexts, a "plausible-sounding" but incorrect pesticide recommendation can lead to crop death or environmental damage.
\end{itemize}

\section{Review of Multi-Modal and Foundation Models}

The current State-of-the-Art (SOTA) in agricultural perception is represented by the move toward multimodal foundation models. A technical overview of these models' adaptation to the agricultural domain is given in this section, with particular attention to grounded reasoning, open-set detection, and semantic verification.
%3.3.1 Limitations of Supervised Detectors (YOLOv8, SSD) regarding unseen classes.
%
%3.3.2 Limitations of Text-Only LLMs regarding visual context (Hallucination).
\subsection{Open-Set Object Detection (Grounding DINO)}
Grounding DINO \cite{liu2023grounding} has recently emerged as a solution for agricultural tasks where traditional closed-set annotations are difficult to acquire. Recent studies have explored its application in diverse field scenarios:
\begin{itemize}
	\item \textbf{Few-Shot Adaptation:} Research by researchers on agricultural datasets (e.g., GWHD, PhenoBench) demonstrates that Grounding DINO can achieve up to 24\% higher mAP than fully fine-tuned YOLO models under few-shot conditions \cite{agri_dino_2024}. This proves its ability to localize complex features like wheat heads and insect pests with minimal training data.
	\item \textbf{Livestock Monitoring:} In cattle management, Grounding DINO has been utilized for muzzle detection without any task-specific training, achieving a mean Average Precision (mAP@0.5) of 76.8\%. This highlights the model's robustness in varied environmental backgrounds.
	\item \textbf{Auto-Labeling Teacher Models:} Some frameworks employ Grounding DINO to automatically generate labels for smaller, real-time models (like YOLO), significantly reducing the manual annotation burden in agricultural computer vision.
\end{itemize}

\subsection{Contrastive Vision-Language Verification (CLIP)}
To address the precision-recall trade-off in open-set detection, researchers are increasingly utilizing CLIP \cite{radford2021learning} as a semantic filter. 
\begin{itemize}
	\item \textbf{Multimodal Disease Classification:} Studies utilizing the PlantVillage and FieldPlant datasets have demonstrated that augmenting visual models with CLIP's textual annotations leads to notable gains in F1-score over vision-only baselines \cite{clip_plant_2024}. These systems use textual descriptions of symptoms (e.g., "irregular brown lesions with yellow halos") to provide a more sophisticated interpretation of the visual data.
	\item \textbf{Few-Shot Frameworks (VLCD):} The Vision-Language model for Crop Disease (VLCD) framework integrates CLIP to improve few-shot classification accuracy. By using a "Cache Model" to blend CLIP's prior knowledge with task-specific training features, researchers have achieved high accuracy in identifying crop leaf diseases with very few samples.
\end{itemize}

\subsection{Retrieval-Augmented Generation (RAG)}
The most recent evolution in agricultural decision support is the deployment of Agentic RAG systems to mitigate the hallucination risks of standard LLMs.
\begin{itemize}
	\item \textbf{AgroLLM and Expert Chatbots:} Frameworks like AgroLLM \cite{agrollm2024} deliver contextually relevant responses derived from a comprehensive corpus of agricultural textbooks and scientific articles. This "non-parametric" grounding ensures that the advice provided is factually accurate and geographically relevant.
	\item \textbf{Q\&A Systems for Pest Management:} Recent Q\&A systems developed for crop pests use adaptive retrieval to handle diverse farmer queries. By combining knowledge graphs with RAG, these systems can refine search queries to retrieve specific pesticide protocols and outbreak patterns, significantly reducing the probability of erroneous diagnostic advice.
\end{itemize}



%3.4.1 Existing Visual Question Answering (VQA) systems in Agronomy.
%
%3.4.2 Recent applications of Foundation Models (DINO/SAM) in agriculture.

%%%%%%%%%%%%%%%%%%%%%%%%%

\section{Research Gap}
The primary research gap identified in this study is the \textbf{lack of an orchestrated agentic workflow} that unifies perception, verification, and reasoning. Specifically:
\begin{enumerate}
	\item \textbf{Absence of Semantic Verification:} While open-set detectors like Grounding DINO offer high recall, there is a lack of research into using contrastive models (CLIP) as a real-time Verifier to suppress the false positives inherent in unstructured field data.
	\item \textbf{The Perception-Reasoning Disconnect:} There is a distinct absence of Agentic RAG implementations where a vision system's output directly initializes a scientific retrieval process without human intervention.
	\item \textbf{Lack of Actionable Autonomy:} Most current models provide a \textit{label} (e.g., ``Tomato Leaf Mold''), but they do not provide a \textit{verified treatment plan} grounded in real-time localized knowledge.
\end{enumerate}

To bridge these gaps, this thesis proposes an ``Autonomous Multimodal Agronomy Agent''. By moving away from the individual assumption of single-task models, this work establishes a unified loop of \textit{Observe $\rightarrow$ Verify $\rightarrow$ Reason $\rightarrow$ Act}. This architectural method is essential for transitioning from simple disease detection to autonomous crop health management.

%%%%%%%%%%%%%%%%%%%%%%%%%



%\section{Review of Multi-Modal and Foundation Models}
%
%To address the bottlenecks of closed-set supervision, the field has transitioned toward Vision-Language Models (VLMs) that utilize shared latent spaces. This section reviews the three pillars of the proposed agent: open-set perception, contrastive verification, and retrieval-augmented reasoning.
%
%\subsection{Open-Set Object Detection (Grounding DINO)}
%Grounding DINO \cite{liu2023grounding} extends the DINO architecture by incorporating language-guided queries. The core technical innovation is the \textbf{Feature Enhancer}, which performs "Neck Fusion" between visual and textual modalities. Unlike standard detectors that use a fixed classification head, Grounding DINO employs \textbf{Cross-Modality Attention} to align image features $I$ with text features $T$.
%
%Mathematically, for a query $Q$, key $K$, and value $V$, the attention mechanism is defined as:
%\[ \text{Attention}(Q, K, V) = \text{softmax}\left(\frac{QK^T}{\sqrt{d_k}}\right)V \]
%
%In the feature enhancer, the visual features $V_{vis}$ are updated by attending to the language features $T_{lang}$:
%\[ V_{vis}^{'} = \text{Cross-Attn}(Q=V_{vis}, K=T_{lang}, V=T_{lang}) \]
%
%This allows the model to localize objects dynamically based on the input string, effectively solving the "Open-Set" problem by treating detection as a phrase-grounding task.
%
%\subsection{Contrastive Verification (CLIP)}
%While Grounding DINO serves as a powerful observer, its open-vocabulary nature can lead to high-recall but low-precision detections. To mitigate this, Contrastive Language-Image Pre-training (CLIP) \cite{radford2021learning} is employed as a semantic filter. CLIP is trained using a \textbf{Contrastive Loss} (InfoNCE) that maximizes the cosine similarity of $N$ correct pairs $(I_i, T_i)$ in a batch while minimizing it for $N^2 - N$ incorrect pairs.
%
%For a batch of image embeddings $i$ and text embeddings $t$, the symmetric cross-entropy loss $L$ is formulated as:
%\[ L = -\frac{1}{2} \left[ \sum_{i=1}^{N} \log \frac{\exp(\text{sim}(I_i, T_i)/\tau)}{\sum_{j=1}^{N} \exp(\text{sim}(I_i, T_j)/\tau)} + \sum_{i=1}^{N} \log \frac{\exp(\text{sim}(I_i, T_i)/\tau)}{\sum_{j=1}^{N} \exp(\text{sim}(T_i, I_j)/\tau)} \right] \]
%
%Where $\tau$ is a learnable temperature parameter. In our system, this allows the agent to verify if a "detected" pest visually aligns with the botanical definition of that pest, effectively acting as a "zero-shot verifier" that requires no domain-specific retraining.
%
%\subsection{Retrieval-Augmented Generation (RAG)}
%The reasoning layer of the agent utilizes Retrieval-Augmented Generation \cite{lewis2020retrieval} to ground the output of the Large Language Model. RAG separates knowledge into two components:
%\begin{enumerate}
%	\item \textbf{Parametric Memory:} The internal weights of the LLM (e.g., Llama-3 or GPT-4).
%	\item \textbf{Non-Parametric Memory:} A vector database (e.g., ChromaDB) containing verified agronomic documents $D$.
%\end{enumerate}
%
%Given a detection $c$, the system retrieves the top-$k$ relevant documents from $D$ based on the embedding similarity $\text{sim}(\phi(c), \phi(D))$. The final diagnosis $y$ is generated by conditioning the model on both the visual context and the retrieved facts:
%\[ y \sim P(y | I, c, D_{k}) \]
%
%This architecture effectively bridges the "Semantic Gap" and ensures that the agent's advice is verifiable and grounded in scientific literature rather than speculative generation.
	\chapter{Methodology}
\label{chap:method}
\thispagestyle{open}
\section{System Overview}
The "Observe → Verify → Reason → Act" Loop.

High-level architectural diagram.

\section{Module A: Adaptive Perception (The Vision Pipeline)}

%4.2.1 Image Pre-processing: Laplacian Variance for blur detection and brightness gating.
%
%4.2.2 Adaptive Inference: Logic for triggering SAHI (Slicing) vs. Standard Inference based on resolution.
%
%4.2.3 Prompt Engineering Strategy: Designing robust ontology prompts (e.g., "tomato leaf . caterpillar . beetle").
%
%4.2.4 The Verification Layer: Using CLIP to filter False Positives (The "Semantic Filter").

\section{Module B: Knowledge Retrieval (The RAG Pipeline)}
%4.3.1 Knowledge Base Construction: Sourcing IPM Manuals and PDFs.
%
%4.3.2 Chunking and Indexing Strategy in ChromaDB.
%
%4.3.3 Query Expansion: Transforming visual tags into semantic search queries.

\section{Module C: Orchestration (The Agent)}

%4.4.1 Pydantic AI Implementation: Defining the State Schema and Type-Safe JSON outputs.
%
%4.4.2 Decision Logic: Routing rules (e.g., differentiating "Benign Bug" from "Pest").
	\chapter{Dataset}
\label{chap:data}
\thispagestyle{open}
\section{Description}

\section{Data Processing}
	\chapter{Prototype Development}
\label{chap:app}
\thispagestyle{open}
%\section{User Requirements}
%
%\section{System Architecture}

\section{Prototype}
The framework is made publicly available on \href{https://github.com/dirtyg2120/multimodal-crop-agent}{https://github.com/dirtyg2120/multimodal-crop-agent}.

The website’s homepage is very simple, just a place for user to upload an image.
\begin{figure}[H]
	\centering
	\frame{\includegraphics[width=\textwidth]{images/1-main.png}}
	\caption{Main page}
	\label{fig:main}
\end{figure}

The main page after user upload an image. There are 3 custom fields: ``caption, tex\_threshold and box\_threshold'' to custom the parameters for running Grouding DINO.

\begin{figure}[H]
	\centering
	\frame{\includegraphics[width=\textwidth]{images/2-upload.png}}
	\caption{Main page after uploading image}
	\label{fig:upload}
\end{figure}

After user click ``Analyze Image'', the model will be applied and return labels with bounding box as shown. 

\begin{figure}[H]
\centering
\frame{\includegraphics[width=\textwidth]{images/3-gdino.png}}
\caption{Visualize the output result from running Grouding DINO}
\label{fig:gdino}
\end{figure}

After Grounding DINO detect all the objects, the system will crop all the bounding boxes and send to CLIP verifier.

\begin{figure}[H]
\centering
\frame{\includegraphics[width=\textwidth]{images/4-result.png}}
\caption{Result show after all steps (Run in the background)}
\label{fig:result}
\end{figure}

If user click on the ``Details'' expander of ``Invidual Leaf Inspection'' part. It will display all cropped images and their labels

\begin{figure}[H]
	\centering
	\frame{\includegraphics[width=\textwidth]{images/6-details.png}}
	\caption{Additional pages to see the label of each object}
	\label{fig:details}
\end{figure}

If user click on the ``Input'' expander of ``Agronomist Diagnosis'' part. It will display the input which will be send to AI Agent, and the part of retrieving RAG is run in the background.

\begin{figure}[H]
\centering
\frame{\includegraphics[width=\textwidth]{images/5-input.png}}
\caption{Additional pages to see the input for AI Agent}
\label{fig:input}
\end{figure}

%%%%%%%%%%%%%%%%%%%%%%%%%

\section{Technologies}
\subsection{PyTorch}
PyTorch is a popular open-source machine learning library primarily used for computer vision and natural language processing. In this project, PyTorch acts as the engine that runs our vision-related models. I use it to load and execute pre-trained models from the Hugging Face library, such as Grounding DINO for detecting objects and CLIP for verifying them.

The main reason for choosing PyTorch is its flexibility and its ability to perform high-speed calculations on a GPU. This is essential for our agent because processing high-resolution images of crops requires significant computing power. By using PyTorch, the agent can analyze images quickly. It also provides a wide range of tools for handling image data, making it easier to integrate the vision system with the rest of the software stack.

\subsection{Pydantic AI}
Pydantic AI is a specialized framework designed to manage how AI agents behave and make decisions. It serves as the ``brain''that controls the flow of our ``Observe-Verify-Reason-Act''loop. While many other tools focus only on sending prompts to a model, Pydantic AI ensures that every piece of information moving through the system follows a strict format.

I use this framework to define exactly what the input and output should look like for each step. For example, when the vision module identifies a disease, Pydantic AI checks that the data is sent in the correct structure (such as a specific list of names and confidence scores) before passing it to the reasoning module. This strict data validation prevents the agent from making errors caused by messy or unexpected information. It makes the system more reliable because the AI is forced to follow a predictable path, reducing the chance of ``hallucinations or logical mistakes.

\subsection{LlamaIndex}
LlamaIndex is a data framework used to connect our Large Language Model (LLM) with external data sources. This process is known as Retrieval-Augmented Generation (RAG). In this project, the agent needs more than just general knowledge; it needs specific, expert-level information about tomato diseases, pests, and treatment protocols.

LlamaIndex allows us to take large documents, such as agricultural textbooks and diagnostic guides, and turn them into a format the agent can easily search. When the agent identifies a specific problem on a plant, LlamaIndex finds the most relevant sections of text from our knowledge base and provides them to the LLM. This ensures that the advice the agent gives to a farmer is grounded in real agricultural science rather than just a guess. It handles the difficult task of indexing and retrieving text so that the agent always has the right context at the right time.

\subsection{ChromaDB}
ChromaDB is a vector database that works alongside LlamaIndex to store and retrieve information based on its meaning. Traditional databases search for exact words, but ChromaDB stores data as embeddings, which are mathematical representations of the meaning of the text. This allows the agent to perform a ``semantic search''.

For instance, if the agent detects ``early blight'' but the documentation uses slightly different terms, ChromaDB can still find the correct information because it understands that the topics are related. I chose ChromaDB because it is lightweight, fast, and easy to set up within a Python environment. It stores all our embedded agricultural data in an organized way, allowing the agent to find the correct treatment protocols in milliseconds. This speed is vital for maintaining a responsive system when the agent has to look through thousands of pages of expert documentation.

\subsection{Streamlit}
Streamlit is an open-source framework used to create the web-based user interface for our application. It allows us to build a professional-looking dashboard using only Python, without needing to write complex HTML, CSS, or JavaScript code.

Through the Streamlit interface, a user can upload a photo of a crop and see the agent's work in progress. The dashboard displays the original image, marks the areas where problems were detected, and shows the final diagnostic report. I chose Streamlit because it is perfect for data-heavy applications. It allows us to visualize the agent's reasoning process in a way that is easy for a human to understand. This transparency is important for building trust with users, as they can see exactly why the agent is making a specific recommendation.

%\subsection{Orchestration \& Backend}
%
%Pydantic AI: Chosen to enforce type safety and structured JSON outputs, preventing the ``stochastic''nature of LLMs from breaking the pipeline.
%
%ChromaDB: Used as the local vector store for low-latency retrieval of agronomic manuals.
%
%\section{User Interface}
%
%Streamlit: Selected for rapid prototyping of the interactive dashboard.
	\chapter{Evaluation Plan and Expected Contributions}
\section{Proposed Evaluation Metrics}
The system will be evaluated using three primary metrics to ensure each part of the loop is performing correctly:
\begin{itemize}
	\item \textbf{Mean Average Precision:} Used to assess the localization accuracy of Grounding DINO in identifying leaves, bugs, and worms.
	\item \textbf{Accuracy:} Used to measure the performance of the dual-CLIP verification layer in identifying specific pathologies.
	\item \textbf{Retrieval Relevance:} Evaluates the RAG system's ability to provide the correct expert treatment advice from ChromaDB based on the identified condition.
	\item \textbf{Visual Case Analysis:} This analysis focuses on the agent's robustness in many other images from the Internet with complex context, ensuring the reasoning and recommendation still perform well.
\end{itemize}

%\chapter{Expected Contributions \& Conclusion}
\section{Expected Contributions}
This thesis aims to contribute to the field of Agentic AI in agriculture by:
\begin{itemize}
	\item Providing a solid ``Observe-Verify-Reason-Act'' framework that bridges the gap between raw vision and expert reasoning.
	\item Demonstrating the use of Open-Set Detection to handle the diverse and changing nature of farm environments without constant retraining.
	\item Developing a method for knowledge grounding that ensures AI recommendations are based on verified agricultural facts.
\end{itemize}

The primary contribution of this research is the development of a stateful agentic loop that integrates open-set detection with expert knowledge. By moving away from static pipelines, this work demonstrates a more flexible and reliable approach to precision agriculture.

\section{Future Work}
Future developments for this project include:
\begin{itemize}
	\item Image Quality Check: Adding a layer to detect motion blur, poor lighting, or extreme occlusion. If the quality is too low, the agent will request a new image rather than risk a wrong diagnosis.
	\item SAHI Integration: Combining Slicing Aided Hyper Inference (SAHI) with these quality checks to help Grounding DINO handle high-resolution images, ensuring small pests are not missed due to image downscaling.
	\item Stateful Memory: The system will save every detection result (with timestamps) into a long-term database. The agent will be able to query this history to identify patterns.
	\item Specialist Model Fallback: Implementing a multi-tier verification system for high-uncertainty cases.
\end{itemize}
	
	% \begin{center}
		% \textbf{\emph{To anyone reading this. Thank you for your time.}}
		% \end{center}
	
	%\backmatter
	%\nocite{*}
	\pagestyle{plain}
	\bibliography{ref}
	\bibliographystyle{plain}
	
\end{document} 