\chapter{Prototype Development}
\label{chap:app}
\thispagestyle{open}
\section{User Requirements}

\section{System Architecture}

\section{Technologies}
%Technologies and Libraries
%
\subsection{Core AI Models}
%
%Grounding DINO & SAHI: Used for zero-shot detection and small object slicing.
%
%CLIP (OpenAI): Employed for the semantic verification layer.
%
%GPT-4o: Selected as the reasoning engine due to its superior instruction-following capabilities compared to smaller open-source models.
%
\subsection{Orchestration \& Backend}
%
%Pydantic AI: Chosen to enforce type safety and structured JSON outputs, preventing the "stochastic" nature of LLMs from breaking the pipeline.
%
%ChromaDB: Used as the local vector store for low-latency retrieval of agronomic manuals.
%
\subsection{User Interface}
%
%Streamlit: Selected for rapid prototyping of the interactive dashboard.